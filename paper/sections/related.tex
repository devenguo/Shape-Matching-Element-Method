\section{Related Work}

%% Set the scene
Geometric modeling is a necessary precursor to physics-based animation, however connecting differing geometric representations for modelling and simulation often requires
time-consuming, complex geometry processing pipelines. 
For instance, the popular tetrahedral finite element approach for simulating solid elastodynamics~\cite{10.1145/2343483.2343501} requires robust algorithms for converting input surface geometry into 
a volumetric tetrahedral mesh. This is a difficult problem and while significant progress has been made,
even the most robust volumetric methods~\cite{Hu:2018:TMW:3197517.3201353} can be time consuming, fail to maintain correspondence between the input model and output simulation mesh,
and don't work directly on curved surface representations such as NURBS.  

%% Outline desiderata for success
%- no meshing of any kind, no tet, hex cut cells, nothing
%- simulation can be directly imported and edited in modelling software
%- supports standard continuum mechanics constitutive models
%- supports different time integrators (can show a few maybe linearly implicit, fully implicit and RK4 ?)

For many physics-based animation tasks, it would be desirable to bypass volumetric meshing entirely and directly simulate the geometric model.
An ideal approach would avoid meshing of any kind (no volumetric meshes or cut-cells), support continuum mechanics-type constitutive models 
and energies that have become standard in physics-based animation pipelines, be compatible with a wide range of time integration schemes and
ensure that simulation output can be edited in the same modelling software used to create the input (important for post-processing). 
Finally, our method should put only moderate constraints on input model quality to facilitate ease-of-use.

\emph{Isogeometric Analysis}~\cite{10.1002/9780470749081.ch7} endeavors to perform simulation directly on the NURBS output from 
Computer-Aided Design (CAD) software. Initial attempts used NURBS surfaces to represent the medial surface of thin objects~\cite{10.1145/176579.176580}.
Volumetric simulations relied on volumetric NURBS~\cite{10.1007/978-3-642-03596-8_2} but were limited to a narrow class of geometries.
Finite volume methods are applicable to more general geometries~\cite{HEINRICH20121645,10.1002/nme.2311} but require a volumetric mesh to be generated.
Modern approaches are constructed around the extended finite element method which enriches the standard finite element basis with discontinuous basis functions
to improve boundary handling~\cite{https://doi.org/10.1002/nme.3120,10.1007/s00466-013-0854-7,https://doi.org/10.1002/nme.4852,SAFDARI2016373,10.1145/3355089.3356576}. 
These methods typically start with an easy-to-generate structured volumetric mesh (tetrahedral or hexahedral), ``cutting'' the NURBS geometric model against
it to enable boundary handling (such a mesh is called a \emph{cut-cell} mesh). 
Like volumetric meshing, this cutting operation can be difficult and our ideal method would avoid it. 
Some cut-cell algorithms assume engineering/manufacturing quality input, which puts tight requirements on input models~\cite{10.1145/3355089.3356576}.
Finally these approaches require additional mechanisms to ensure that simulation results lie inside the shape space of the input model's primitives,
which increases the complexity. 

\revise{
\emph{Embedded Methods} attempt to sidestep many of these issues by enclosing complex surface geometry inside a simulation coarse mesh~\cite{10.5555/1009379.1009573}. 
However, producing an appropriate mesh embedding introduces a number of challenges since the coarse mesh's connectivity must mirror that of the embedded surface.
This necessitates the use of complex hierarchical methods like that of~\citet{10.1145/1531326.1531358}, which themselves require the user to correctly intuit an appropriate ultimate grid resolution.
Alternately one can introduce cut cell~\cite{Tao:Mandoline:2019} or extended Finite Element~\cite{10.1145/3355089.3356576} approaches which bring with them complicated geometric operations. 
Furthermore when the DOFs do not lie on the boundary, methods such as Nitsche's Method ~\cite{Nitsche1971berEV} are required for handling Dirichlet boundary conditions. 
Our method inherits shape connectivity directly from the boundary input, avoiding these difficulties entirely.
}

\emph{Shape Matching} is a meshless approach to physics-based animation~\cite{10.1145/1073204.1073216,10.1145/2019406.2019438} and geometry processing~\cite{10.1111/j.1467-8659.2012.03171.x}
built around shape registration. The algorithm has been extended from volumetric triangle mesh input to cloth~\cite{10.2312/PE/vriphys/vriphys08/039-046}, to particles~\cite{10.1145/1964921.1964987}
and even to visual geometry for video games~\cite{10.1145/2994258.2994260}. 
Shape Matching is fast~\cite{10.1145/1275808.1276480,10.5555/1632592.1632606} and meshless, but state-of-the-art methods require 
additional modelling input to position simulation primitives~\cite{10.1145/1964921.1964987} or limit simulation primitives to be 
collections of convex polytopes~\cite{10.1145/2994258.2994260}.
Finally, Shape Matching is tightly coupled to the position-based dynamics (PBD)~\cite{muller2007position}
time integration methodology. 
While incredibly performant, this approach is incompatible with the constitutive models that are popular for physics-based animation, as well as 
other time integration schemes. 
The popular Projective Dynamics algorithm~\cite{10.1145/2601097.2601116} enables a more flexible Shape Matching implementation but is still limited to a subset of
constitutive models for elastic solids. 

To alleviate these restrictions we turn to other meshless methods popular in computer graphics and 
engineering~\cite{10.1145/1028523.1028542,https://doi.org/10.1002/fld.1650200824,10.1145/1778765.1778776,10.1145/2010324.1964968, 10.1145/1944846.1944855}.
These methods support more advanced constitutive models and integration schemes but often require background integration meshes, limit themselves to low order deformation functions 
and lose the direct connection with
modelling geometry. 
Like cut-cell-based, Isogeometric Analysis approaches, additional constraints must be added to ensure that simulation output can be represented by the input model.
Given this, we conclude that there is no existing algorithm that meets our desiderata for success. 

Our algorithm takes the Shape Matching approach as inspiration, but rather than follow the PBD formalism, we interpret Shape Matching as a 
Virtual Element Method (VEM)~\cite{10.1142/S0218202512500492,10.1142/S021820251440003X}. Virtual Elements are an extension of 
mimetic finite differences~\cite{10.1142/S0218202505000832,10.1016/j.jcp.2013.07.031} to weak-form variational problems. 
VEM relaxes the mesh generation requirements of the finite element method by enabling the solution of partial differential equations
on domains partitioned with arbitrary polytopes~\cite{10.1145/3386569.3392389}. 
The solution inside each polytope is approximated using a polynomial function of a specified order.
VEM typically assumes that the boundary of the the problem domain is described by a piecewise linear complex
which makes its standard formulation incompatible with our curved input geometry. 


\subsection{Contributions} 
In this paper we develop a new Shape Matching-based, Virtual Element Method which is directly compatible 
with NURBS input geometry rather than piecewise linear surfaces. 
Furthermore, we improve the expressivity of the VEM basis by using a shape blending approach inspired by
algorithms for skinning~\cite{skinningcourse:2014}.
Our method is entirely meshless (requiring no volumetric meshes or cut-cells) and is compatible with standard constitutive models and time integrators.
It guarantees that simulation output is directly consumable by the input modelling software and can ingest models which include large gaps,
intersections and disconnected primitives without additional user input. 
These features mean that our algorithm is the first truly meshless approach for the direct simulation of curved surface, NURBS models in physics-based animation.

%\subsection{Virtual Element and other Element Methods}
%\begin{enumerate}
    %\item \textit{Mimetic Finite Differences} \cite{10.1142/S0218202505000832} \cite{10.1016/j.jcp.2013.07.031} (they double dipped!): Considered a close relative to VEM and framed as the predecessor to VEM (in VEM papers). I still haven't read on MFD yet. From PolyDDF: "extension of finite volume and finite difference techniques to polygons that first discretizes a prime operator (typically, the gradient or the divergence) via a boundary integral, and then derives other operators by mimicking continuous structural properties."
    %\item \textit{Basic principles of Virtual Element Method} \cite{10.1142/S0218202512500492} Original VEM paper
    %\item \textit{The Hitchhiker's Guide to Virtual Element Method} \cite{10.1142/S021820251440003X}: More understandable versions of original VEM paper.
    %\item \textit{Discrete Differential Operators on Polygonal Meshes} \cite{10.1145/3386569.3392389}: Extends VEM to do discrete differential geometry on arbitrary polygonal meshes.
    %Don't think we need to cite this \item \textit{FLexible Simulation of Deformable Models using Discontinuous Galerkin FEM} \cite{10.5555/1632592.1632608}: Uses ordinary hexahedral elements, but uses a cut cell-based approach to support arbitrary polyhedra on the surface. 
    %Later ?\item \textit{Generalizing the finite element method: Diffuse approximation and diffuse elements} \cite{Nayroles1992GeneralizingTF} Predecessor to Element Free Galerkin. FEM interpolation replaced with a local Moving Least Square interpolation.
    %Later ?\item \textit{Element-free Galerkin methods} \cite{https://doi.org/10.1002/nme.1620370205}: similar to DEM, but more accurate gradients (not sure of all the differences). In MLS methods, they solve least squares for each particle in the domain, weighting nearby particles with a Gaussian-like density function. In contrast to us, we only compute least squares fitting on the boundary, and then precompute some weighting for each particle to the projection operators on the boundaries.
    %\item \textit{Unified Simulation of Elastic Rods, Shells, and Solids} \cite{10.1145/1778765.1778776}: Propose Generalized moving least squares (GMLS) to resolve limitation of MLS shape functions that require many particles in the support of a point (that are not coplonar).  
%\end{enumerate}


%\subsection{Shape Matching Related Papers}
%\begin{enumerate}
%    \item \textit{Shape Matching} \cite{10.1145/1073204.1073216}: Meshless simulation by fitting a polynomial describing the shapes deformation using only the nodal values.
    %\item \textit{Lattice Shape Matching} \cite{10.1145/1275808.1276480}: Voxelize model to construct a lattice of cubes. Use these lattice cubes to construct overlapping shape matching regions (just like clustered shape matching?). The original mesh is deformed using trilinear interpolation of lattice vertex positions.
    %\item \textit{Robust Real-Time Deformation of Incompressible Surface Meshes} \cite{10.1145/2019406.2019438}: Shape matching on trimeshes with overlapping regions (clustered shape matching). Adds an additional volume preservation constraint. Position based dynamics approach to satisfying volume preservation.
 %   \item \textit{Shape-Up: Shaping Discrete Geometry with Projections} \cite{10.1111/j.1467-8659.2012.03171.x}: Shape constraints by least squares fitting (like in shape matching). They have some "proximity function" indicating distance to least-squares fit, then uses projection operators to minimize proximity function (pretty much just shape matching). 
    %\item \textit{Shape Matching with Oriented Particles} \cite{10.1145/1964921.1964987}: More general form for shape matching, permitting wider range of motion. Also they use shape matching projection operators for skinning, much like we do. For each skinning point, they specify weights with up to 4 particles (each with their own projection operator)
    %\item \textit{Fast Adaptive Shape Matching Deformations} \cite{10.5555/1632592.1632606}: Essentially same thing as Lattice Shape Matching, but instead they use an octree instead of a basic voxel grid for shape matching. It's not super significant to mention this paper, but it does make clear that much of the followup work after shape matching never didn't emphasize it's utility as a meshless boundary only method. They kept converting it to a mesh-based method!
    %\item \textit{A Geometric Deformation Model for Stable Cloth Simulation} \cite{10.2312/PE/vriphys/vriphys08/039-046} Shape matching for cloth simulation.
%\end{enumerate}
%% start with iso geometric analysis (XCAD here)
%\subsection{Isogeometric Analysis}
%\begin{enumerate}
    %\item \textit{Isogeometric Analysis Book} \cite{10.1002/9780470749081.ch7}: The book everyone references when they write Isogeometric analysis in their papers.
    %\item \textit{Dynamic NURBS} \cite{10.1145/176579.176580}: Outline of how to simulate on NURBS with the control points as the degrees of freedom. Method used in our work.
    %\item \textit{XCAD} \cite{10.1145/3355089.3356576}: Optimize CAD models. CAD embedded in hexahedral mesh, complex integration strategy. Uncut hexahedral elements simulated ordinarily, cut elements use XFEM that add additional DOF to account for new element shapes.
    %\item \textit{Development of a quadratic finite element formulation based on the XFEM and NURBS} \cite{https://doi.org/10.1002/nme.3120}: XFEM to handle curve surface integration of NURBs patches. Complex subdividing of "X-Elements" to produce cut cells along NURBS surfaces.
    %\item \textit{A NURBS enhanced extended finite element approach for unfitted CAD analysis} \cite{10.1007/s00466-013-0854-7}: Pretty much the same as the quadratic, but allows higher-order approximation and better handling of interface.
    %\item \textit{A NURBS-based interface-enriched generalized finite element method for problems with complex discontinuous gradient fields} \cite{https://doi.org/10.1002/nme.4852}: Similar to other XFEM NURBS approaches. Uses NURBS-based enrichment functions with cut cells. Additional DOFs added to handle discontinuities.
    %\item \textit{A NURBS-based generalized finite element scheme for 3D simulation of heterogeneous materials} \cite{SAFDARI2016373}: Similar to previous "NIGFEM" paper above, but now in 3D.
    %\item \textit{Swept Volume Parameterization for Isogeometric} \cite{10.1007/978-3-642-03596-8_2} To provide volumetric simulation of NURBS they introduce a new NURBS volume parameterization (B-Spline Volumes ... jesus christ)
    %\item \textit{A finite volume method on NURBS geometries and its application in isogeometric fluid–structure interaction} \cite{HEINRICH20121645}: Combines NURBS paramaterization with finite volume method (requiring a mesh for the volume). 
    %\item \textit{NURBS-Enhanced Finite Element Method (NEFEM)} \cite{10.1002/nme.2311}: Similar to the above example. They run an order FVM simulation and deal with the interface in a complicated manner (could this be considered XFEM?).
%\end{enumerate}


%% start with standard fem stuff
%% start with iso geometric analysis (XCAD here)
%% talk about shape matching and simulating visual geometry but not constitutive models
%% talk about how sparse mesh free methods fixes these things but still loses connection to modelling geometry
%% - ^^^ throw rig space physics in here talk about constraints talk about finding method that does this by construction but is still meshes.
%% talk about VEM has a potential solution 
%% limitation is that it looks at piecewise linear stuff, one polynomial per element, limits expressivity




%\subsection{Other Meshless Methods in Graphics}
%\begin{enumerate}
    %\item \textit{Point Based Animation }\cite{10.1145/1028523.1028542}: Purely particle based, MLS to approximate derivatives. Appears to be among the earliest meshless methods in graphics based on continuum mechanics. 
    %\item \textit{Position Based Dynamics} \cite{MULLER2007109}: Operates directly on particle positions by forming set of constraints and solving for particle positions that satisfy these constraints. Meshless
    %\item \textit{Projective Dynamics} \cite{10.1145/2601097.2601116}: Similar to position based dynamics but solves the constraints implicitly by minimizing energy potentials. Mesh-based. Global solver unlike PBD which satisfies constraints locally using Gauss-Seidel.
%\end{enumerate}

%%% Might need later in methods section
%\subsection{Physics based Skinning}
%\begin{enumerate}
%    \item \textit{Skinning Siggraph Course} \cite{10.1145/2614028.2615427}: The linear weighting of polynomials at the exterior is very similar to skinning.
%    \item \textit{Linear Subspace Design for Real-Time Shape Deformation} \cite{10.1145/2766952}: Linear deformation subspace that uses linear blend skinning and generalized barycentric coordinates. Similarly, we have "handles" but they are represented by entire NURBS patches, and our coordinates are the output of a polynomial, whereas barycentric coordinates are linear (I only skimmed this paper, not sure if this description is fair).
%    \item \textit{Complementary dynamics} \cite{Zhang_2020}: Physics based skinning, orthogonality constraint can be seen as similar to our stability term (conformity term, error term, whatever it's called :))
%    \item \textit{Physically-Based Character Skinning} \cite{10.2312/PE.vriphys.vriphys13.025-034}: Linear blend skinning with multiple layers of skin simulated via oriented particle shape matching and position based dynamics to enforce distance constraints (avoiding unwanted intersections).
%\end{enumerate}

%Might need later in methods
%\subsection{Quadrature}
%\begin{enumerate}
%    \item \textit{A new method for meshless integration in 2D and 3D Galerkin meshfree methods} \cite{KHOSRAVIFARD201030}: Strategy we use for integrating over CAD model volumes. Raycast along single dimension, find intersections, generate quadrature points in the intervals inside the object.
%    \item Adaptive image-based intersection \cite{DBLP:journals/tog/WangFP12}: related to our meshless integration strategy in that we could use this to account for errors in the above approach (increase ray density where we estimate high error to be)
%    \item Efficient and accurate numerical quadrature for immersed boundary methods \cite{10.1186/s40323-015-0031-y}: Finite Cell Method. "Immerses" a shape in a set of cells (mesh!) and computes quadrature over this. To handle curved surface they use an octree to subdivide to the desired level of accuracy.
%    \item Higher-Order Finite Elements for Embedded Simulation \cite{10.1145/3414685.3417853}: Another Finite Cell method like the above, but with a new quadrature generation method (the ones with the circles in the triangles)
%    \item \textit{Highly accurate surface and volume integration on implicit domains by means of moment-fitting} \cite{https://doi.org/10.1002/nme.4569} and \cite{https://doi.org/10.1002/nme.5343}: XCAD paper extends upon this method \myworries{still need to read these}
%\end{enumerate}

%Might need later
%\subsection{Misc}
%\begin{enumerate}
%	\item TRACKS: Toward Directable Thin Shells \cite{10.1145/1276377.1276439}: Petrov-Galerkin test functions for weak-form constraints that handles artifacts due to pointwise constraints.
    %\item Fusion 360 Gallery \cite{willis2020fusion} : source of some models
%\end{enumerate}