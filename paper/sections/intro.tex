\section{Introduction}

The consumption of geometric surface models by physics-based animation algorithms is fraught with difficulty. 
For volumetric objects, this process often involves identifying and discretizing the interior of the modelled object, 
typically either as a tetrahedral or hexahedral mesh. 
This procedure is both expensive and difficult, especially if the surface model is constructed from higher-order boundary
representations, or if the volumetric discretization is required to be conforming or feature aligned. 
Removing explicit volumetric discretization from the physics-based-animation pipeline can avoid these difficulties and 
also provide a more unified modelling and simulation experience. 

NURBS (Non-uniform Rational B Splines) are a popular higher-order modelling primitive which 
are used for computer-aided design (CAD), computational fabrication and computer animation. 
NURBS primitives were the first geometric representation used for physics-based animation~\cite{10.1145/37402.37427}, yet,
despite over three decades of research, animation of curved models such as those built with NURBS, remains a challenge.

Isogeometric Analysis (IGA) is a physics simulation methodology that uses the control variables of the NURBS model
as the degrees-of-freedom (DOFs) of the simulation itself. 
Unfortunately IGA approaches for volumetric objects still require background volumetric structures, typically
regular grids that make satisfaction of boundary conditions difficult (which makes collision resolution hard)
or more complicated cut-cell grids which introduce non-trivial root finding problems into the mix. 
Crucially, these simulation schemes typically assume models arise from engineering applications and meet tight geometric
criteria such as that the mesh is manufacturable. 
These inputs are much cleaner than those produced by a typical animation modeller. 

% In this paper we present a new finite element method for isogeometric, physics-based animation of NURBs models.
% Crucially our method is boundary only, and requires no volumetric meshes (thus avoiding their inherent complications), grids or cutcells of any kind
% Our method does not need an explicit labelling of the inside/outide of the simulated model
% Does not require nurbs patches in an object to be explicitly joined
% Input is just the raw NURBs model with no additional information or annotations
% The resulting simulation algorithm is compatiable with nonlinear continuum mechanics material models as well as 
% standard methods for contact resolution. 

We present the first truly meshless (no volumetric discretization required) algorithm for direct, nonlinear elastodynamic simulation of NURBs models. 
Our nonlinear elastodynamic simulation scheme requires only a boundary description of the object (we do not strictly require a solid model) and
approriate physical parameters.  
Because we explicitly use the NURBS boundary representation in the simulation, it is straightforward to handle
Dirichlet and Neumann boundary conditions and to apply contact resolution. 
Crucially, because we broadly target animation not necessarily simulation for engineering or manufacturing, we don't require that models
satisfy the rigourous geometric requirements common for these applications.

Our approach draws inspiration from the recently developed Virtual Element Method (VEM) for solving partial differential equations on domains tiled with arbitrary polygons.
We establish a connection between VEM and the well-known shape matching simulation algorithm~\cite{10.1145/1073204.1073216,10.1145/2019406.2019438}. This allows us 
to derive equations of motion for an arbitrary curved surface model, made of NURBS, via Lagrangian Mechanics. 
Importantly, we show how to replace volumetric data structures for integration and blending weight computation with ray casting approaches which enables
our meshless approach to elastodynamic physically-based animation.

% Modelling -> Animation -> Rendering
% Use physics for animation which requires conversion from modelling to simulation representation
% for volumetric objects this conversion invovles identififyin and meshing the interior of the modelled object and that's hard
% especially difficult if object surface is described by higher-order boundary representations. 
% Ideally we would have a more direct method of translating our surface model into a volumetric, simulatable form
%
% NURBs  are an oft-used higher order modelling primitive popular for CAD and Computer Graphics for which nice modelling tools exist
% Rhinocerous 3D and Fusion 360.
% Despite lots of work performing physics-based animation on NURBS models is hard
% robustly uilding a volumetric polygonal mesh that simulataneously maintains a corresonpondence to the model surface is an open-research problem
% Isogeometric approaches which directly use the NURBs primitives as simulation variables require additional background grids, complex cutcell algorithms
% and often must rely on soft boundary conditions.
%
% Want to say there is an advantage to avoiding constructing volumetric representations because it costs time and adds complexity
% that;s our goal to provide an alternative, the first truly meshless approach to IGA which maintains a tightt mapping to the boundary
% In this paper we present a new finite element method for isogeometric, physics-based animation of NURBs models.
% Crucially our method is boundary only, and requires no volumetric meshes (thus avoiding their inherent complications), grids or cutcells of any kind
% Our method does not need an explicit labelling of the inside/outide of the simulated model
% Does not require nurbs patches in an object to be explicitly joined
% Input is just the raw NURBs model with no additional information or annotations
% The resulting simulation algorithm is compatiable with nonlinear continuum mechanics material models as well as 
% standard methods for contact resolution. 
%
% crucially because we broadly target animation not necessarily simulation for engineering or manufacturing we don't require that models
% satisfy rigourous geometric requirements common for these applications
%
% Our approach relies on the recently developed Virtual Element Method for solving PDEs on domains tiled with arbitrary polygons.
% By establsihing the connection between VEM and the well-known shape matching simulation algorithm we derive equations of motion sfor 
% an arbitrary NURBS model using Lagrangian Mechanics. We avoid using volumetric data structures for integration by demonstrating that 
% raycasting provides a fast and reliable alternative. In combination, our contributions amount to the first algorithm for 
% isogeometric elastodynamics of volumetric structures that is truly meshless (introducing no auxilillary volumetric data structures) 
% which serves to close the considerable gap betweeen modelling and physics-based animation. 
%
% Desiderata for a sucessful algorithm for PBA 
% No volumetric mesh creation
% Support for Nonlinear Constitutive Models
% Support for standard Collision 
% DOFS on the Boundary (useful for boundary conditions, contact)
% Automatic determination of model volume
% Non-manufacturable models 
% Code Available 

%limitations
% sometimes you want some volumetric dofs to better resolve internal behaviour