\section{Future Work and Conclusions}
We have presented the Shape Matching Element Method (SEM), the first completely meshless approach to direct simulation of curved surface models, made from NURBS primitives.
Our approach is unique in its ability to infer volumetric shape from surface only input,
including input with intersecting geometry between parts and other defects common to non-engineering models.
We believe that SEM is a significant improvement over standard physics-based animation pipelines. 
As evidence of this, the authors submit that many of the examples in this paper were constructed ourselves
(since the standard graphics menagerie is not available as NURBS models). 
Modelling, cleaning, meshing and simulating would've been a burdensome experience without SEM's
ability to leap directly from (often hastily) constructed models to physics-based animations.

SEM, as it's presented here, is in its infancy and we believe there are many exciting areas of future work to explore.
We are very excited to couple SEM to machine learning approaches for design, parameter estimation and Real2Sim applications.
One of the cumbersome elements in using finite element simulation for such problems is the need for robust, differentiable volumetric meshing~\cite{gao2020deftet}.
SEM removes this bottleneck entirely, providing a direct mapping from geometric input to physics-driven output. 
We believe SEM will enable simpler and more robust algorithms for  physics-based ML and 
allow the application of such algorithms to a much broader class of problems. 

SEM itself has much room for improvement. 
First, while we focus on NURBS surfaces here, the only part of SEM that is NURBS specific is the shape matching operation.
We believe there is potential to allow mixed models (models which include polygonal meshes, particles, subdivision surfaces and NURBS)
by extending the range of shape matching operations used by the algorithm. The shape matching operation itself could be improved to be material-aware 
(to better handle heterogenous materials) or to be robust to noisy data (allowing direct simulation of scanned data). 
It is also interesting to consider the relationship between our error term and the orthogonality constraint proposed by~\citet{Zhang:CompDynamics:2020}. 
Their approach could be used to remove this parameter from SEM entirely.
Finally, in this paper we have focused on applying SEM to physics-based animation, and there is a significant amount of additional work needed to 
extend SEM reliably into engineering applications.

The Finite Element method took over 40 years to mature to its current state and to become the preeminent tool in physics-based animation.
We hope that this is the beginning of a similar, exciting journey for SEM. 
Motivated by this sentiment, and to encourage future research on SEM, the authors will release our SEM implementation 
under a permissive license as well as all models created for this submission.
